\documentclass{article}

\usepackage{extramarks}

\usepackage{amsmath}
\usepackage{amsthm}
\usepackage{amssymb}
\usepackage{amsfonts}
\usepackage{caption}
\usepackage{subcaption}
\usepackage{listings}

\usepackage{hyperref}

\usepackage{tikz}

\usepackage{algorithm}
\usepackage{algorithmic}
\usepackage[shortlabels]{enumitem}

\usepackage{float,graphicx}

\usepackage{pgfplots}

\usepackage{adjustbox}


\usepackage{array}
\usepackage{pgfgantt}

\usepackage[utf8]{inputenc}

\usepackage{fancyhdr}
\usepackage{fancybox}

\topmargin=-0.45in
\evensidemargin=0in
\oddsidemargin=0in
\textwidth=6.5in
\textheight=9.0in
\headsep=0.25in

% Listings' Styles

\definecolor{codegreen}{rgb}{0,0.6,0}
\definecolor{codegray}{rgb}{0.5,0.5,0.5}
\definecolor{codepurple}{rgb}{0.58,0,0.82}
\definecolor{backcolour}{rgb}{0.96,0.96,0.96}


\lstdefinestyle{python}{
    backgroundcolor=\color{backcolour},
    commentstyle=\color{codegreen},
    keywordstyle=\color{magenta},
    numberstyle=\tiny\color{codegray},
    stringstyle=\color{codepurple},
    basicstyle=\footnotesize,
    breakatwhitespace=false,
    breaklines=true,
    captionpos=b,
    keepspaces=true,
    numbers=left,                    
    numbersep=5pt,                  
    showspaces=false,                
    showstringspaces=false,
    showtabs=false,                  
    tabsize=2
}
 
\lstset{style=python}
\lstset{language=Python}

\linespread{1.1}
\captionsetup[table]{position=bottom}

\pagestyle{fancy}
\lhead{\groupName}
\chead{\hmwkClass: \hmwkTitle}
\rhead{\today}
\lfoot{\lastxmark}
\cfoot{\thepage}

\renewcommand\headrulewidth{0.4pt}
\renewcommand\footrulewidth{0.4pt}
\renewcommand\qedsymbol{$\blacksquare$}

\setlength\parindent{0pt}

\newcommand{\hmwkTitle}{Assignment Sheet\ \#\hmwkNumber}
\newcommand{\hmwkDueDate}{February 22, 2022}
\newcommand{\hmwkClass}{Machine Learning}
\newcommand{\hmwkAuthorName}{Henri Sota, Enis Mustafaj}
\newcommand{\groupName}{\textbf{Group HB}}

% Homework Number Variable
\newcommand{\hmwkNumber}{2}

%
% Create Problem Sections
%

\newcommand{\enterProblemHeader}[1]{
    \nobreak{}
}

\newcommand{\exitProblemHeader}[1]{
    \stepcounter{#1}
}

\setcounter{secnumdepth}{0}
\newcounter{partCounter}
\newcounter{homeworkProblemCounter}

\setcounter{homeworkProblemCounter}{1}
\nobreak\extramarks{Problem \hmwkNumber \arabic{homeworkProblemCounter}}{}\nobreak{}

%
% Homework Problem Environment
%
% This environment takes an optional argument. When given, it will adjust the
% problem counter. This is useful for when the problems given for your
% assignment aren't sequential. See the last 3 problems of this template for an
% example.
%
\newenvironment{homeworkProblem}[1][-1]{
    \ifnum#1>0
        \setcounter{homeworkProblemCounter}{\hmwkNumber.#1}
    \fi
    \section{Problem \hmwkNumber.\arabic{homeworkProblemCounter}}
    \setcounter{partCounter}{1}
    \enterProblemHeader{homeworkProblemCounter}
}{
    \exitProblemHeader{homeworkProblemCounter}
}


\newcounter{programmingPartCounter}
\newcounter{programmingProblemCounter}

\setcounter{programmingProblemCounter}{1}
\nobreak\extramarks{Programming Problem \hmwkNumber \arabic{programmingProblemCounter}}{}\nobreak{}

%
% Programming Problem Environment
%
% This environment takes an optional argument. When given, it will adjust the
% problem counter. This is useful for when the problems given for your
% assignment aren't sequential. See the last 3 problems of this template for an
% example.
%
\newenvironment{programmingProblem}[1][-1]{
    \ifnum#1>0
        \setcounter{programmingProblemCounter}{\hmwkNumber.#1}
    \fi
    \section{Programming Problem \hmwkNumber.\arabic{programmingProblemCounter}}
    \setcounter{programmingPartCounter}{1}
    \enterProblemHeader{programmingProblemCounter}
}{
    \exitProblemHeader{programmingProblemCounter}
}

\title{
    \vspace{2in}
    \textmd{\textbf{\hmwkClass:\\ \hmwkTitle}}\\
    \normalsize\vspace{0.1in}\small{Due\ on\ \hmwkDueDate\ at 10:00}\\
    \vspace{3in}
}

\author{\groupName \\ \hmwkAuthorName}
\date{}

\renewcommand{\part}[1]{\textbf{\large Part \Alph{partCounter}}\stepcounter{partCounter}\\}

% Useful for algorithms
\newcommand{\alg}[1]{\textsc{\bfseries \footnotesize #1}}

% Alias for the Solution section header
\newcommand{\solution}{\textbf{\large Solution}}
\newcommand{\comment}[1]{} % Multi-line comment


\begin{document}
\maketitle
\pagebreak
\begin{homeworkProblem}
In this exercise, you work with two data sets:
\begin{itemize}
    \item \href{https://archive.ics.uci.edu/ml/datasets/Statlog+\%28Shuttle\%29}{Statlog (Shuttle) Data Set}
    \item \href{https://archive.ics.uci.edu/ml/datasets/Computer+Hardware}{Computer Hardware Data Set}
\end{itemize}
which are both available in the \href{https://archive.ics.uci.edu/ml/datasets.php}{UCI Machine Learning Repository}. For each of them, perform the following tasks:

\begin{enumerate}[a)]
    \item Briefly describe the data set and all involved variables in your own words. If some information is missing on the UCI Repository site, do your own search for these details.
    \item Model the data set via input and output random variables / vectors.
    \item Formulate a question that can be solved using machine learning on this data set and give the type of machine learning (supervised / unsupervised / regression / classification) that will allow to answer the question.
\end{enumerate}
\end{homeworkProblem}

\begin{homeworkProblem}
The \href{https://archive.ics.uci.edu/ml/datasets/spambase}{Spambase Data Set} is a SPAM classification data set that has exactly the 57 input variables that are roughly described in Example 2.9 of the lecture.

\begin{enumerate}[a)]
    \item In Example 2.9, we did not mention the specific words and characters that are used in the features. Use the information of the UCI Repository to make a complete description of these variables, i.e. give all the key words, etc.
    \item Search the web for alternative features that can be used to describe (SPAM) emails. Pick one example feature, cite the source, and describe these features.
\end{enumerate}
\end{homeworkProblem}

\begin{programmingProblem}
Implement a code that reads a text file and extracts the 57 features discussed in the previous task. Apply the feature extractor to the provided three example emails.
\end{programmingProblem}

\end{document}
