\documentclass{article}

\usepackage{extramarks}

\usepackage{amsmath}
\usepackage{amsthm}
\usepackage{amssymb}
\usepackage{amsfonts}
\usepackage{caption}
\usepackage{subcaption}
\usepackage{listings}

\usepackage{hyperref}

\usepackage{tikz}

\usepackage{algorithm}
\usepackage{algorithmic}
\usepackage[shortlabels]{enumitem}

\usepackage{float,graphicx}

\usepackage{pgfplots}

\usepackage{adjustbox}


\usepackage{array}
\usepackage{pgfgantt}

\usepackage[utf8]{inputenc}

\usepackage{fancyhdr}
\usepackage{fancybox}

\topmargin=-0.45in
\evensidemargin=0in
\oddsidemargin=0in
\textwidth=6.5in
\textheight=9.0in
\headsep=0.25in

% Listings' Styles

\definecolor{codegreen}{rgb}{0,0.6,0}
\definecolor{codegray}{rgb}{0.5,0.5,0.5}
\definecolor{codepurple}{rgb}{0.58,0,0.82}
\definecolor{backcolour}{rgb}{0.96,0.96,0.96}


\lstdefinestyle{python}{
    backgroundcolor=\color{backcolour},
    commentstyle=\color{codegreen},
    keywordstyle=\color{magenta},
    numberstyle=\tiny\color{codegray},
    stringstyle=\color{codepurple},
    basicstyle=\footnotesize,
    breakatwhitespace=false,
    breaklines=true,
    captionpos=b,
    keepspaces=true,
    numbers=left,
    numbersep=5pt,
    showspaces=false,
    showstringspaces=false,
    showtabs=false,
    tabsize=2
}

\lstset{style=python}
\lstset{language=Python}

\linespread{1.1}
\captionsetup[table]{position=bottom}

\pagestyle{fancy}
\lhead{\groupName}
\chead{\hmwkClass: \hmwkTitle}
\rhead{\today}
\lfoot{\lastxmark}
\cfoot{\thepage}

\renewcommand\headrulewidth{0.4pt}
\renewcommand\footrulewidth{0.4pt}
\renewcommand\qedsymbol{$\blacksquare$}

\setlength\parindent{0pt}

\newcommand{\hmwkTitle}{Assignment Sheet\ \#\hmwkNumber}
\newcommand{\hmwkDueDate}{April 5, 2022}
\newcommand{\hmwkClass}{Machine Learning}
\newcommand{\hmwkAuthorName}{Henri Sota, Enis Mustafaj}
\newcommand{\groupName}{\textbf{Group HB}}

% Homework Number Variable
\newcommand{\hmwkNumber}{7}

%
% Create Problem Sections
%

\newcommand{\enterProblemHeader}[1]{
    \nobreak{}
}

\newcommand{\exitProblemHeader}[1]{
    \stepcounter{#1}
}

\setcounter{secnumdepth}{0}
\newcounter{partCounter}
\newcounter{homeworkProblemCounter}

\setcounter{homeworkProblemCounter}{1}
\nobreak\extramarks{Problem \hmwkNumber \arabic{homeworkProblemCounter}}{}\nobreak{}

%
% Homework Problem Environment
%
% This environment takes an optional argument. When given, it will adjust the
% problem counter. This is useful for when the problems given for your
% assignment aren't sequential. See the last 3 problems of this template for an
% example.
%
\newenvironment{homeworkProblem}[1][-1]{
    \ifnum#1>0
        \setcounter{homeworkProblemCounter}{\hmwkNumber.#1}
    \fi
    \section{Problem \hmwkNumber.\arabic{homeworkProblemCounter}}
    \setcounter{partCounter}{1}
    \enterProblemHeader{homeworkProblemCounter}
}{
    \exitProblemHeader{homeworkProblemCounter}
}


\newcounter{programmingPartCounter}
\newcounter{programmingProblemCounter}

\setcounter{programmingProblemCounter}{1}
\nobreak\extramarks{Programming Problem \hmwkNumber \arabic{programmingProblemCounter}}{}\nobreak{}

%
% Programming Problem Environment
%
% This environment takes an optional argument. When given, it will adjust the
% problem counter. This is useful for when the problems given for your
% assignment aren't sequential. See the last 3 problems of this template for an
% example.
%
\newenvironment{programmingProblem}[1][-1]{
    \ifnum#1>0
        \setcounter{programmingProblemCounter}{\hmwkNumber.#1}
    \fi
    \section{Programming Problem \hmwkNumber.\arabic{programmingProblemCounter}}
    \setcounter{programmingPartCounter}{1}
    \enterProblemHeader{programmingProblemCounter}
}{
    \exitProblemHeader{programmingProblemCounter}
}

\title{
    \vspace{2in}
    \textmd{\textbf{\hmwkClass:\\ \hmwkTitle}}\\
    \normalsize\vspace{0.1in}\small{Due\ on\ \hmwkDueDate\ at 10:00}\\
    \vspace{3in}
}

\author{\groupName \\ \hmwkAuthorName}
\date{}

\renewcommand{\part}[1]{\textbf{\large Part \Alph{partCounter}}\stepcounter{partCounter}\\}

% Useful for algorithms
\newcommand{\alg}[1]{\textsc{\bfseries \footnotesize #1}}

% Alias for the Solution section header
\newcommand{\solution}{\textbf{\large Solution}}
\newcommand{\comment}[1]{} % Multi-line comment

\DeclareMathOperator*{\argmax}{arg\,max}
\DeclareMathOperator*{\argmin}{arg\,min}

\begin{document}
\maketitle
\pagebreak
\begin{homeworkProblem}
In this task, we consider the training data
\begin{equation*}
    \resizebox{\textwidth}{!} {
    $\mathcal{T} = \left \{ ((1, 7)^{\top}, 25), ((2, 5)^{\top}, 21), ((2, 6)^{\top}, 14), ((3, 3)^{\top}, 32), ((7, 1)^{\top}, 14), ((3, 1)^{\top}, 14), ((4, 2)^{\top}, 25), ((5, 4)^{\top}, 18), ((5, 6)^{\top}, 12) \right \}$
    }
\end{equation*}
\begin{enumerate}[a)]
    \item Build a predictor using kNN regression for $k = 6$ and evaluate the training error of the predictor.
    \item Build a predictor using linear regression by least squares and evaluate the training error of the linear model.
\end{enumerate}
\end{homeworkProblem}
\begin{homeworkProblem}
In this task, we consider the data set
\begin{equation*}
    \mathcal{T} = \left \{ (-2, 4), (2, 4), (1, 1), (-1, 1), (0, 0), (3, 9) \right \}
\end{equation*}
\begin{enumerate}[a)]
    \item Evaluate the (expected) generalization error of the kNN regressor with $k = 2$ by $\mathcal{K}$-fold cross validation with $\mathcal{K} = 3$. (Do a deterministic, i.e. not randomized, splitting of the given data following the ordering of the samples.)
    \item Evaluate the (expected) generalization error of the linear model by leave-one-out cross validation. (It is fine to use the help of a computer to fit the individual linear models.)
\end{enumerate}
\end{homeworkProblem}
\begin{homeworkProblem}
Provide quantitative reasoning to the questions below.
\begin{enumerate}[a)]
    \item What is the training error of kNN regression with neighbourhood size $k = 1$? Give the result and an explanation of how you get to the result.
    \item What is the training error of linear regression with an output dimension of $K = 1$ for $N = 2$ distinct training samples? Give the result and an explanation of how you get to the result.
\end{enumerate}    
\end{homeworkProblem}
\begin{programmingProblem}
In this programming exercise, you will implement the validation set approach and $\mathcal{K}$-fold cross validation, while recalling that leave-one-out cross validation is $\mathcal{K}$-fold cross validation for $\mathcal{K} = N$, if $N$ is the number of training samples.\\
Implement the study carried out in Example 6.3 from the lecture notes. Note that due to randomization, you might get other results that the ones shown in the example.
\end{programmingProblem}

\end{document}
